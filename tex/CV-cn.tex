% !TEX program = xelatex
% This file is generated, don't manually edit!
\documentclass{resume}
\usepackage{lastpage}
\usepackage{fancyhdr}
\usepackage{linespacing_fix}
\usepackage[fallback]{xeCJK}
\begin{document}
\renewcommand\headrulewidth{0pt}
\name{朱明志}
\basicInfo{
\email{zmzfpc@tongji.edu.cn}\textperiodcentered\
\github[zmzfpc]{https://github.com/zmzfpc}\textperiodcentered\
\phone{+8619946254050}
}
\section{教育经历}
\datedsubsection{\textbf{同济大学}, 中国}{2018.09 --现在}
专业:软件工程(本科),预计毕业日期:2022.06
\section{项目经历}
\datedsubsection{\textbf{同济大学交通运输工程学院\&嘉定公交公司
},上海,中国}{11/2020 -- 05/2021}
\role{公交驾驶员行为感知和安全监测系统}{核心代码开发,成员}
\begin{itemize}[parsep=0.25ex]
	\item 使用Keras框架和已有Inception模型组件搭建深度神经网络,用于驾驶员动作感知。
	\item 利用Adam\&RMSprop两个优化器分步对模型进行训练并设置Dropout层等优化训练结果,减少过拟合。
	\item 使用激活类图的技术对模型感知结果可视化输出。
	\item 利用决策树模型,结合OCR等技术识别出视频中的时空信息,对驾驶员行为的安全性进行评估,对不安全的行为进行报警。
	\item 基于React框架,开发了一个安全监测系统,基于上述训练完成的模型,对驾驶员不当行为进行预警、记录和介入干涉。
\end{itemize}

\datedsubsection{\textbf{同济大学软件学院
	},上海,中国}{09/2020 -- 01/2021}
\role{CodePass云端代码问答社区}{前端开发,组长}
\begin{itemize}[parsep=0.25ex]
	\item 基于React框架开发前端代码,使用ant-design图标库设计前端的UI界面,对整个前端的界面风格和配词进行统筹设计。
	\item 设计前端界面的跳转逻辑,并设计了登陆信息弹出的逻辑和请求状态弹出的逻辑。
	\item 开发了问题点赞评论功能,设置了用户点赞一次后再点赞取消点赞的功能,以及按时间先后顺序展示用户评论功能。
\end{itemize}

\datedsubsection{\textbf{COMAP
	},远程}{02/2021}
\role{MCM 亚洲大黄蜂入侵}{MaxEnT模型训练及推送规则制定,组长}
\begin{itemize}[parsep=0.25ex]
	\item 使用MaxEnt生态位模型,结合美国华盛顿州的一些GIS数据,利用已知的大黄蜂入侵信息作为标签,有监督的训练生态位模型,利用最大熵的方法预测某地发生大黄蜂入侵的概率。
	\item 对训练的生态位模型进行评估,使用响应曲线的方法,评估每一个地理环境变量对预测结果的影响指数,并绘制ROC曲线评估模型预测的合理性。
	\item 利用drools制定了一个推送规则,按照这个规则向研究人员推送民众提交的报告,优先推送高可靠报告。
\end{itemize}
\datedsubsection{\textbf{同济大学电子与信息工程学院
	},上海,中国}{04/2021 -- 现在}
\role{金融网络风险评估}{代码开发,第一负责人}
\begin{itemize}[parsep=0.25ex]
	\item 基于R语言NetworkRiskMeasures包提供的数据格式,利用基于内容的智能爬虫爬取世界15万家银行的3项核心数据。
	\item 改进NetworkRiskMeasures包的银行内拆借网络模拟算法,并使用Python对其进行重构,以期提高生成模拟网络的速率。
\end{itemize}
\section{技能}
\begin{itemize}[parsep=0.25ex]
\item \textbf{编程语言}:\textbf{多种语言},
熟练使用C++、C、Python、Tex和JavaScript,熟悉Java、C\#、R、HTML、CSS.(排名不分先后)
\item \textbf{机器学习}:有分类和聚类问题的实战经验,对分类和聚类算法的原理有一定了解。
\item \textbf{深度学习}:熟练使用Keras框架,环境背景熟悉TensorFlow2.x,对神经网络的原理有一定理解,有ANN、CNN网络的搭建经验。
\item \textbf{前端开发}:有React和Vue框架的使用经验,有开发React Webapp的经验。
\item \textbf{开发工具}:能适应任何编辑器/操作系统,平常在Windows下使用VSCode和JetBrains IDE,有使用GitHub、Gitee团队协作工具的经验。
\end{itemize}
\section{其它}
\begin{itemize}[parsep=0.25ex]
	\item 2021美国大学生数学建模竞赛,\url{https://www.comap-math.com/mcm/2021Certs/2123864.pdf}, Meritorious Winner
	\item 2020全国大学生竞赛(非数学类),二等奖
	\item 2019同济大学校一等奖学金
	%\item IntelliJ Plugin developer profile: \url{https://plugins.jetbrains.com/author/10a216dd-c558-4aaf-aa8a-723f431452fb}
	
\end{itemize}
\end{document}
